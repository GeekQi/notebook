% !TeX spellcheck = en_US

%----------------------------------------------------------------------------------------
%	PACKAGES AND OTHER DOCUMENT CONFIGURATIONS
%----------------------------------------------------------------------------------------

\XeTeXlinebreaklocale "zh"
\XeTeXlinebreakskip = 0pt plus 1pt
\documentclass[
10pt, % Main document font size
a4paper, % Paper type, use 'letterpaper' for US Letter paper
oneside, % One page layout (no page indentation)
%twoside, % Two page layout (page indentation for binding and different headers)
headinclude,footinclude, % Extra spacing for the header and footer
BCOR5mm, % Binding correction
]{scrartcl}
\input{structure.tex} % Include the structure.tex file which specified the document structure and layout

\hyphenation{Fortran hy-phen-ation} % Specify custom hyphenation points in words with dashes where you would like hyphenation to occur, or alternatively, don't put any dashes in a word to stop hyphenation altogether

%----------------------------------------------------------------------------------------
%	TITLE AND AUTHOR(S)
%----------------------------------------------------------------------------------------
\usepackage{fontspec}
\usepackage{xcolor}
%\setmainfont{STKaiti}   
\title{\normalfont\spacedallcaps{I don't care}} % The article title

\author{\spacedlowsmallcaps{I am yangqi}} % The article author(s) - author affiliations need to be specified in the AUTHOR AFFILIATIONS block

\date{} % An optional date to appear under the author(s)

%----------------------------------------------------------------------------------------
\usepackage{titletoc} 
\setmainfont{STKaiti} 
\setsansfont{STKaiti}
\setmonofont{STKaiti}

%\setmainfont[BoldFont=STHeiti]{STSong}
%\setsansfont[BoldFont=Adobe Heiti Std]{STKaiti}}
\begin{document}
%----------------------------------------------------------------------------------------
%	HEADERS
%----------------------------------------------------------------------------------------
%\renewcommand{\sectionmark}[1]{\markright{\spacedlowsmallcaps{#1}}} % The header for all pages (oneside) or for even pages (twoside)
%\renewcommand{\subsectionmark}[1]{\markright{\thesubsection~#1}} % Uncomment when using the twoside option - this modifies the header on odd pages
%\lehead{\mbox{\llap{\small\thepage\kern1em\color{halfgray} \vline}\color{halfgray}\hspace{0.5em}\rightmark\hfil}} % The header style

%\pagestyle{scrheadings} % Enable the headers specified in this block

%----------------------------------------------------------------------------------------
%	TABLE OF CONTENTS & LISTS OF FIGURES AND TABLES
%----------------------------------------------------------------------------------------

\maketitle % Print the title/author/date block
%
\setcounter{tocdepth}{3} % Set the depth of the table of contents to show sections and subsections only
%
\tableofcontents % Print the table of contents
%
%\listoffigures % Print the list of figures
%
%\listoftables % Print the list of tables

%----------------------------------------------------------------------------------------
%	ABSTRACT
%----------------------------------------------------------------------------------------

%\section*{Abstract} % This section will not appear in the table of contents due to the star (\section*)

%\lipsum[1] % Dummy text

%----------------------------------------------------------------------------------------
%	AUTHOR AFFILIATIONS
%----------------------------------------------------------------------------------------

%{\let\thefootnote\relax\footnotetext{* \textit{Department of Biology, University of Examples, London, United Kingdom}}}

%{\let\thefootnote\relax\footnotetext{\textsuperscript{1} \textit{Department of Chemistry, University of Examples, London, United Kingdom}}}

%----------------------------------------------------------------------------------------

\newpage % Start the article content on the second page, remove this if you have a longer abstract that goes onto the second page


%----------------------------------------------------------------------------------------
%	NLP-Blog
%----------------------------------------------------------------------------------------


\section{NLP-Blog}
关键词:\\\indent 自然语言处理(NLP).词向量(Word Vectors).奇异值分解(Singular Value Decomposition). Skip-gram. 连续词袋(CBOW),负采样样本(Negative Sampling)
%A statement\footnote{Example of a footnote} requiring citation \cite{Figueredo:2009dg}.
\subsection{\color{red}词向量}
What: 词组用向量表示
\\Why:\\\indent1.NLP转为ML问题,第一步就是将符号数学化\\\indent2词向量编码词组,使其代表N维空间中的一个点,点与点之间距离可以代表深层信息。每一个词向量的维度都可能会表征一些意义(物理含义)。例如,语义维度可以用来表明时态(过去与现在与未来),计数(单数与复数),和性别(男性与女性)
\\How:编码方式:比如one-hot vector
\subsubsection{\color{blue}one-hot vector}
What:对词库中n个词,每个词在某个index下取到1,其余位置为0
\\Disadvantge:\\\indent1.维数灾难\\\indent2.词向量无法表示词组相似性: (W$^{hotel}$)$^T$w$^{motel}$ = (W$^{hotel}$)$^T$w$^{cat}$ = 0[hotel和motel是近义词]。
\\Improve:可以把词向量的维度降低一些,在这样一个子空间中,可能原本没有关联的词就关联起来了
\subsection{\color{red}构造词向量方法-基于SVD}
How:遍历所有的文本数据集,然后统计词出现的次数,接着用一个矩阵X来表示所有的次数情况,紧接着对X进行奇异值分解得到一个USV$^T$的分解。然后用U的行(rows)作为所有词表中词的词向量。对于矩阵X,有如下方法:
\subsubsection{\color{blue}词-文档矩阵}
What:行:文档M。列:词组V。
\\How:遍历文件,词组i出现在文件j中,将Xij值加一。得到矩阵R|V|×M
\subsubsection{\color{blue}基于窗口的共现矩阵}
What:同上,将词频换成了相关性矩阵
\\How: 固定大小窗口,统计每个词出现在窗口中次数。
\\\indent 例如: I enjoy flying. || I like NLP. || I like deep learning. 
\begin{figure}[!htb]
	\centering
	\includegraphics[width=0.5\linewidth]{screenshot003}
	\caption{基于窗口共现矩阵}
	\label{fig:screenshot003}
\end{figure}
\subsubsection{\color{blue}奇异值分解构造词向量矩阵}
What:将矩阵用更小更简单的子矩阵的相乘来表示【机器学会抽取重要特征】
\\Why:降维
\\Example:PCA、数据(图像)压缩、搜索引擎语义层次检索LSI
\\How: \\\indent A = U$\Sigma$V$^T$ 并根据保留百分比保留k个维度.\indent【奇异值σ跟特征值类似,在矩阵Σ中也是从大到小排列,而且σ的减少特别的快,在很多情况下,前10\%甚至1\%的奇异值的和就占了全部的奇异值之和的99\%以上了。也就是说,我们也可以用前r大的奇异值来近似描述矩阵】
\\Result: \\\indent U作为词嵌入矩阵,对于词表中的每一个词,都用一个k维的向量表示
\\Disadvantage:\\\indent 1.矩阵稀疏,纬度高且经常变化\\\indent 2.复杂度:$O(n^2)$(SVD).且需要对矩阵X处理
\subsection{\color{red}构造词向量方法-基于迭代的方法}
Why:为了解决奇异值分解矩阵的问题
What:可以一步步迭代学习的模型,最终得到每个单词基于上下文的条件概率
\subsubsection{\color{blue}语言模型(1-gram, 2-gram)}
一元模型:
\begin{displaymath}
P(w_1, w_2, w_3...w_n) = \prod_{i=1}^{N}P(w_i)\\
\end{displaymath}
\\ 二元模型:
\begin{displaymath}
P(w_1, w_2, w_3...w_n) = \prod_{i=1}^{N}P(w_i | w_{i-1})
\end{displaymath}
\\Disadvantage:只考虑一个词语依赖相邻词语的关系
\subsubsection{\color{blue}连续词袋模型CBOW-语言模型的进步}
What:以句子其他部分为上下文,预测或产生中心词语。
\\How:【如下图】
\\\indent 1.模型已知参数:将句子表示为一些onehot向量: x(c)
\\\indent 2.模型输出:y(c)[其实只有一个输出]
\\\indent 3.模型未知参数:两矩阵:V $\in R^{n*|V|}$ 和 U $\in R^{|V|*n}$ ,n任意。
\\\indent \indent V:输入词矩阵. 词语$w_i$作为输入时,V的第i列$v_i$为$w_i$的输入向量
\\\indent \indent U:输出词矩阵. 词语$w_j$作为输出时,V的第j列$v_j$为$w_j$的输出向量
\\\indent 4.模型过程:
\\\indent \indent a.产生onehot向量$(x^{c-m}...x^{c-1},x^{c+1}...x^{c+m})$
\\\indent \indent b.得到上下文嵌入词向量(输入):
\\\indent \indent $(v_{c-m+1} = Vx^{(c-m+1)},...,v_{c+m} = Vx^{(c+m)})$
\\\indent \indent c.向量取平均:$\hat{v}=\frac{2m}{v_{c-m}+v_{c-m+1}+...+v_{c+m}}$
\\\indent \indent d.得到得分向量:z = U$\hat{v}$并转化成概率分布形式$\hat{y}=softmax(z)$
\\\indent \indent e.我们希望产生的概率分布与期望的真实词语的onehot向量相匹配
\\\\如何找到U、V--- 需要有一个目标函数
\\What:评估差异/损失的函数y与$\hat{y}$--交叉熵$H(\hat{y}, y)= -y_i log(\hat{y_i})$
\\How: 通过如下优化函数,用梯度下降法更新每个相关词向量$u_c$和$v_j$
\begin{figure}[!htb]
	\centering
	\includegraphics[width=0.7\linewidth]{screenshot008}
	\caption*{}
	\label{fig:screenshot008}
\end{figure}
\begin{figure}[!htb]
	\centering
	\includegraphics[width=0.7\linewidth]{screenshot009}
	\caption{左:CBOW模型过程--右:Skip-Gram模型过程}
	\label{fig:screenshot009}
\end{figure}
\subsubsection{\color{blue}Skip-Gram模型-与CBOW相对应}
What:以中心词为输入,预测周围词。则中心词为上下文
\\How:【如上图】
\\\indent 模型过程:
\\\indent \indent a.产生onehot向量x
\\\indent \indent b.得到上下文嵌入词向量(输入):$v_c = Vx$
\\\indent \indent c.向量取平均[直接是自身]:$v_c = v_c$
\\\indent \indent d.通过u=U$v_c$得到2m个得分向量:$u_{c-m},..,u_{c-1},u_{c+1},u{c+m}$
\\\indent \indent e.并转化成概率分布形式$y=softmax(u)$
\\\indent \indent f.我们希望产生概率分布与期望真实概率分布$y^{c-m},..,y^{c-1},y^{c+1},..,y^{c+m}$相匹配
\\\indent 引入朴素贝叶斯假设将联合概率拆分成独立概率相乘【给出中心词,所有输出词完全独立】,再用随机梯度下降更新未知参数。
\begin{figure}[!htb]
	\centering
	\includegraphics[width=0.7\linewidth]{screenshot010}
	\caption*{}
	\label{fig:screenshot010}
\end{figure}
\subsubsection{\color{blue}负面抽样(Negative Samplint)-简化目标函数求值}
What:【参考文献下方】目标函数对整个单词表|V|的求和计算量巨大。要简化,则就近似。
\\How:对每一步训练,不去循环整个单词表,而是抽象一些负面例子。
\\\indent 从噪声分布$P_n(w)$中抽样,其概率分布与单词表中频率相匹配。
\\\indent 则只需要更新:目标函数/梯度/更新规则
\\负面抽样-基于SkipGram模型,但对不同目标函数优化
\\\indent a.对“词-上下文”对(w, c).令P(D=1|w, c)为(w, c)来自语料库的概率
\\\indent a.令P(D=0|w, c)是不来自语料库的概率
\\\indent b.对P(D=1|w, c)用sigmoid函数建模:$P(D=1|w, \theta, c) = \frac{1}{1+e^{(-v_c^Tv_w)}}$
\\\indent c.需要新的目标函数:如果(w, c)来自语料库,目标函数能最大化P(D=1|w, c)
\\\indent d.对这两个概率用最大似然
\begin{figure}[!htb]
	\centering
	\includegraphics[width=0.8\linewidth]{screenshot011}
	\caption*{}
	\label{fig:screenshot011}
\end{figure}
\subsection{词向量的评价}
{\let\thefootnote\relax\footnotetext{ \textit{Mikolov ET AL.《Distributed Representations of Words and Phrases and their Compositionality》}}}

\newpage 
%----------------------------------------------------------------------------------------
%	YQ-Algorithm
%----------------------------------------------------------------------------------------

\section{Algorithm}


%------------------------------------------------

\subsection{算法}

\newpage 
%----------------------------------------------------------------------------------------
%	YQ-Machine-Learning
%----------------------------------------------------------------------------------------
\section{Machine-Learning}
\subsection{\color{red}学习算法}
\subsubsection{\color{blue}监督学习}
What:数据集中每个样本都有“正确答案”,再根据样本作出预测
\\Example:\\\indent 回归问题:\\\indent \indent What:推导出连续的输出。\\\indent \indent Example:房价分析/销量预测
\\\indent 分类问题:\\\indent \indent What:推导出离散的输出。\\\indent \indent Example:乳腺肿瘤判断/垃圾邮件问题
\subsubsection{\color{blue}无监督学习}
What:交给算法大量数据,让算法为我们从数据中找出某种结构
\\Example:\\\indent 聚类问题:\\\indent \indent How:谷歌news。同一主题的聚类
\subsection{\color{red}第一个学习算法-单变量线性回归}
What:只有一个特征(输入变量)
\\\indent 回归:\\\indent \indent What:根据之前的数据预测一个准确输出值
\\\indent 线性回归:\\\indent \indent What: 确定两种或两种以上变量间相互依赖的定量关系。y = w$\'$x + e
\\How:
\begin{figure}[!htb]
	\centering
	\includegraphics[width=0.4\linewidth, height=0.15\textheight]{screenshot004}
	\caption{监督学习算法工作方式}
	\label{fig:screenshot004}
\end{figure}
\\\indent h的一种可能表达方式:h$_\theta$(x)  = $\theta_0$ + $\theta_1$x 因为只有一个特征(输入变量), 这样的问题叫做单变量线性回归问题.
\subsubsection{\color{blue}代价函数-用来求解线性回归方程参数}
What:平方误差函数(平方误差代价函数)。
\\How:建模误差的平方和: J($\theta_0$, $\theta_1$) = $\frac{1}{2m}\sum_{i=1}^m (h_\theta(x^i))-y^i)^2$
\\\indent 目标:Min J($\theta_0$, $\theta_1$)
\\\indent How:【如下图】
\\\indent 1. 代价函数(等高线图):在三维空间中存在一个值使得J($\theta_0$, $\theta_1$)最小
\begin{figure}[!htb]
	\centering
	\includegraphics[width=0.7\linewidth]{screenshot006}
	\caption{代价函数-等高线图}
	\label{fig:screenshot006}
\end{figure}
\\\indent 2. 需要算法【自动】找出使得J 最小化的$\theta_0$, $\theta_1$的值
\subsubsection{\color{blue}梯度下降-求代价函数最小值}
What:求函数最小值的算法
\\How:随机选择一个参数的组合($\theta_0$, $\theta_1$,$\theta_2$, $\theta_3$),计算代价函数,然后寻找下一个能让代价函数值下降最多的参数组合。直到到达一个局部最小值。【由于没有常识所有的参数组合,不能保证局部最小值】



\end{document}